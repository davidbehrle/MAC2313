\documentclass[a4paper]{article}
\usepackage[english]{babel}
\usepackage[utf8]{inputenc}
\usepackage{textcomp}
\usepackage{amsmath}
\usepackage{amssymb}
\usepackage{gensymb}
\usepackage{physics}
\usepackage{graphicx}
\usepackage[colorinlistoftodos]{todonotes}
\usepackage{xcolor}
\usepackage{array}
\usepackage{tabularx}
\usepackage{tikz}
\usepackage{pgfplots}
\usepackage{framed}
\usepackage{xfrac}
\usepackage[most]{tcolorbox}
\usepackage{fix-cm}
\usepackage{cancel}
\usepackage{pagecolor}
\usepackage[margin=0.5in]{geometry}
\usetikzlibrary{quotes,angles}
\usetikzlibrary{decorations.pathreplacing}
\usetikzlibrary{calc}
\usetikzlibrary{external}
\tikzexternalize[prefix=tikz/ch2figures/]
\usepgfplotslibrary{fillbetween}

\let\phi\varphi
\let\bf\textbf
\let\la\langle
\let\ra\rangle
\pgfplotsset{compat=1.18}
\newcommand\der[2]{\frac{d #1}{d #2}}
\newcommand\Deltat{\Delta t}
\newcommand{\ih}{\hat{\imath}}
\newcommand{\jh}{\hat{\jmath}}
\newcommand{\AxisRotator}[1][rotate=0]{%
    \tikz [x=0.25cm,y=0.60cm,line width=.2ex,-stealth,#1] \draw (0,0) arc (-150:150:1 and 1);%
}
\def\centerarc[#1](#2)(#3:#4:#5){\draw[#1] ($(#2)+({#5*cos(#3)},{#5*sin(#3)})$) arc (#3:#4:#5)}
% Syntax: \centerarc[draw options] (center) (initial angle:final angle:radius);

\title{Vectors in Space}
\author{OpenStax Calculus Vol. 3}
\date{}

\definecolor{fg0}{HTML}{fbf1c7}

\makeatletter % change only the display of \thepage, but not \thepage itself:
\patchcmd{\ps@plain}{\thepage}{\textcolor{fg1}{\thepage}}{}{}
\makeatother

\begin{document}
\pagestyle{plain}

\definecolor{bg0}{HTML}{282828}
\definecolor{bg0_h}{HTML}{1d2021}
\definecolor{bg0_s}{HTML}{32302f}
\definecolor{bg1}{HTML}{3c3836}
\definecolor{bg2}{HTML}{504945}
\definecolor{bg3}{HTML}{665c54}
\definecolor{bg4}{HTML}{7c6f64}
\definecolor{fg1}{HTML}{ebdbb2}
\definecolor{fg2}{HTML}{d5c4a1}
\definecolor{fg3}{HTML}{bdae93}
\definecolor{fg4}{HTML}{a89984}
\definecolor{gbaqua}{HTML}{8ec07c}
\definecolor{gbaqua2}{HTML}{689d6a}
\definecolor{gbred}{HTML}{fb4934}
\definecolor{gbred2}{HTML}{cc241d}
\definecolor{gbgreen}{HTML}{b8bb26}
\definecolor{gbgreen2}{HTML}{98971a}
\definecolor{gbyellow}{HTML}{fabd2f}
\definecolor{gbyellow2}{HTML}{d79921}
\definecolor{gbblue}{HTML}{83a598}
\definecolor{gbblue2}{HTML}{458588}
\definecolor{gbpurple}{HTML}{d3869b}
\definecolor{gbpurple2}{HTML}{b16286}
\definecolor{gborange}{HTML}{fe8019}
\definecolor{gborange2}{HTML}{d65d0e}
\definecolor{gbgray}{HTML}{a89984}
\definecolor{gbgray2}{HTML}{928374}
\color{fg0}
\pagecolor{bg0_s}
\colorlet{shadecolor}{gbaqua2}

\setcounter{section}{2}
\maketitle

\subsection{Vectors in the Plane}
A vector in a plane is represented by a directed line section, its endpoints are called the initial and terminal points respectively. An arrow from the initial point to the terminal point indicates the direction of the vector and the length of the line segment represents its magnitude. The notation $||\vec{v}||$ or $||\bf{v}||$ is used to denote the magnitude of the vector $\vec{v}$ (or \bf{v}). The zero vector is a vector with the same initial and terminal point denoted $\vec{0}$ or \bf{0}. 
\vspace{1mm}\\
Vectors with the same magnitude and direction are considered equivalent vectors and are treated as equal even if they have different initial points. Two vectors are parallel if they have the same or opposite directions. Vectors are defined by magnitude and direction regardless of the location of the initial point.
\vspace{2mm}\\
\bf{Combining Vectors}\vspace{2mm}\\
Scalars are quantities that have only a magnitude and no direction. Multiplying a vector by a scalar changes the vector's magnitude; this is called scalar multiplication. Changing the magnitude of a vector does not indicate a change in its direction.
\begin{shaded}
    \noindent\underline{\bf{Definition of Scalar Multiplication}}\vspace{2mm}\\
    The product $k\vec{v}$ of a vector $\vec{v}$ and scalar $k$ is a vector with a magnitude that is $|k| \cdot ||\vec{v}||$ and a direction that is equal to the direction of $\vec{v}$ if $k>0$ and opposite the direction of $\vec{v}$ if $k<0$. If $k = 0$ or $\vec{v} = \vec{0} \implies k\vec{v} = \vec{0}$
\end{shaded}
\noindent Because each vector may have its own direction, the process for adding vectors is different from adding scalars. The most common graphical method for adding two vectors is to place the initial point of the second vector at the terminal point of the first. The sum of the vectors $\vec{v}$ and $\vec{w}$ is the vector with an initial point that coincides with the initial point of $\vec{v}$ and a terminal point that coincides with the terminal point of $\vec{w}$.
\begin{center}
    \begin{tikzpicture}[scale=2]
        \draw (0,0) coordinate (a);
        \draw ({1.5*cos(20)},{1.5*sin(20)}) coordinate (b);
        \draw ({cos(110)},{sin(110)}) coordinate (c);
        \draw ({1.5*cos(20) + cos(110)},{1.5*sin(20) + sin(110)})coordinate (d);
        \draw[dashed,very thick,gbblue2] (c)--(d)--(b);
        \draw[->,very thick,-latex,gbblue2] (a)--node[left,color=fg1,xshift=-0.25em]{$\vec{v}$}(c);
        \draw[->,very thick,-latex,gbblue2] (a)--node[below,color=fg1]{$\vec{w}$}(b);
        \draw[->,very thick,-latex,gbred2] (a)--node[above,color=fg1,rotate=53.69]{$\vec{v} + \vec{w}$}(d);
        \filldraw[fg1] (a) circle (1pt);
        \filldraw[fg1] (b) circle (1pt);
        \filldraw[fg1] (c) circle (1pt);
        \filldraw[fg1] (d) circle (1pt);
    \end{tikzpicture}
    \hspace{1.5cm}
    \begin{tikzpicture}[scale=2]
        \draw (0,0) coordinate (a);
        \draw ({1.5*cos(20)},{1.5*sin(20)}) coordinate (b);
        \draw ({cos(110)},{sin(110)}) coordinate (c);
        \draw ({1.5*cos(20) + cos(110)},{1.5*sin(20) + sin(110)})coordinate (d);
        \draw[dashed,very thick,gbblue2] (d)--(b)--(a);
        \draw[->,very thick,-latex,gbblue2] (a)--node[left,color=fg1,xshift=-0.25em]{$\vec{v}$}(c);
        \draw[->,very thick,-latex,gbblue2] (c)--node[above,color=fg1]{$\vec{w}$}(d);
        \draw[->,very thick,-latex,gbred2] (a)--node[above,color=fg1,rotate=53.69]{$\vec{v} + \vec{w}$}(d);
        \filldraw[fg1] (a) circle (1pt);
        \filldraw[fg1] (b) circle (1pt);
        \filldraw[fg1] (c) circle (1pt);
        \filldraw[fg1] (d) circle (1pt);
    \end{tikzpicture}
\end{center}
For $\vec{u} = \vec{v} + \vec{w}$, the initial point of $\vec{u}$ is the initial point of $\vec{v}$ and the terminal point is the terminal point of $\vec{w}$. These three vectors form a triangle, it follows that the length of any one side is less than the sum of the lengths of the remaining sides, therefore: $||\vec{u}|| \leq ||\vec{v}|| + ||\vec{w}||$
\vspace{1mm}\\
For vector subtraction, $\vec{v} - \vec{w}$ is defined as $\vec{v} + (-\vec{w}) = \vec{v} + (-1)\vec{w}$. The vector $\vec{v} - \vec{w}$ is called the vector difference and is depicted graphically by drawing a vector from the terminal point of $\vec{w}$ to the terminal point of $\vec{v}$.
\begin{center}
    \begin{tikzpicture}[scale=1.5]
        \draw (0,0) coordinate (a);
        \draw ({1.2*cos(20)},{1.2*sin(20)}) coordinate (b);
        \draw ({cos(110)},{sin(110)}) coordinate (c);
        \draw ({1.5*cos(20) + cos(110)},{1.5*sin(20) + sin(110)})coordinate (d);
        \draw[->,very thick,-latex,gbred2] (b)--node[right,color=fg1]{$\vec{v} - \vec{w}$}(d);
        \draw[->,very thick,-latex,gbblue2] (a)--node[below,color=fg1]{$\vec{w}$} (b);
        \draw[->,very thick,-latex,gbblue2] (a)--node[above,color=fg1,xshift=-0.5em]{$\vec{v}$}(d);
    \end{tikzpicture}
    \hspace{1.5cm}
    \begin{tikzpicture}[scale=1.5]
        \draw (0,0) coordinate (a);
        \draw ({1.2*cos(20)},{1.2*sin(20)}) coordinate (b);
        \draw ({(1.5*cos(20) + cos(110)) - (1.2*cos(20))},{(1.5*sin(20) + sin(110)) - (1.2*sin(20))}) coordinate (c);
        \draw ({1.5*cos(20) + cos(110)},{1.5*sin(20) + sin(110)})coordinate (d);
        \draw[dashed,very thick,gbred2] (b)--(d);
        \draw[->,very thick,-latex,gbred2] (a)--node[left,color=fg1]{$\vec{v}+ (-\vec{w})$}(c);
        \draw[dashed,very thick,gbblue2] (a)--(b);
        \draw[->,very thick,-latex,gbblue2] (a)--node[above,color=fg1,xshift=-0.5em]{$\vec{v}$}(d);
        \draw[->,very thick,-latex,gbblue2] (d)--node[above,color=fg1]{$-\vec{w}$}(c);
    \end{tikzpicture}
\end{center}
\bf{Vector Components}\vspace{2mm}\\
A vector with initial point at the origin is called a standard-position vector. Because the initial point of any vector in standard position is (0,0), it can be described by the coordinates of its terminal point.
\begin{shaded}
    \noindent\underline{\bf{Definition}}
    \vspace{2mm}\\
    The vector with initial point (0,0) and terminal point $(x,y)$ can be written in component form as
    \begin{align*}
        \vec{v} = \la x,y \ra
    \end{align*}
    The scalars $x$ and $y$ are called the components of $\vec{v}$
\end{shaded}
\noindent For a vector not already in standard form its component form can be determined algebraically by subtracting the $x$ and $y$ values of the initial point from those of the terminal point.
\begin{shaded}
    \noindent Let $\vec{v}$ be a vector with initial point $(x_i,y_i)$ and terminal point $(x_t,y_t)$. Then $\vec{v}$ can be expressed in component form as 
    \begin{align*}
        \vec{v}= \la x_t - x_i, y_t - y_i \ra
    \end{align*}
\end{shaded}
\noindent To find the magnitude of a vector, calculate the distance between its initial and terminal points. The magnitude of vector $\vec{v} = \la x, y \ra$ is denoted $||\vec{v}||$ and can be calculated using the formula
\begin{align*}
    ||\vec{v}|| = \sqrt{x^2 + y^2}
\end{align*}
Based on this, it is clear that for any vector $\vec{v}$, $||\vec{v}|| \geq 0$, and $||\vec{v}|| = 0$ if and only if $\vec{v} = \vec{0}$.\vspace{1mm}\\
The magnitude of a vector can also be derived using Pythagorean theorem.
\begin{center}
    \begin{tikzpicture}[scale=1.75]
        \draw (0,0) coordinate (a);
        \draw (2,0) coordinate (b);
        \draw (2,0.75) coordinate (c);
        \draw[bg3] (1.85,0)--(1.85,0.15)--(2,0.15);
        \draw[thick,fg1] (a)--node[below,color=fg1]{$x$}(b)--node[right,color=fg1]{$y$}(c);
        \draw[->,very thick,-latex,gborange] (a)--node[above,color=fg1,xshift=-0.75em]{$\sqrt{x^2 + y^2}$} (c); 
    \end{tikzpicture}
\end{center}
Expressing vectors in component form allows scalar multiplication and vector addition to be performed algebraically.
\begin{shaded}
    \noindent Let $\vec{v} = \la x_1, y_1 \ra$ and $\vec{w} = \la x_2,y_2 \ra$ be vectors, and let $k$ be a scalar.\vspace{1mm}\\
    \bf{Scalar Multiplication:} $k\vec{v} = \la kx_1, ky_1 \ra$\\
    \bf{Vector Addition:} $\vec{v} + \vec{w} = \la x_1,y_1 \ra + \la x_2,y_2 \ra = \la x_1 + x_2, y_1 + y_2 \ra$
\end{shaded}

\begin{shaded}
    \noindent\underline{\bf{Properties of Vector Operations}}\vspace{2mm}\\
    Let $\vec{u}, \vec{v}$, and $\vec{w}$ be vectors in a plane. Let $r$ and $s$ be scalars.
    \begin{align*}
        \text{i}&. &\vec{u} + \vec{v} &= \vec{v} + \vec{u} &\text{Commutative Property}\\
        \text{ii}&. &(\vec{u} + \vec{v}) + \vec{w} &= \vec{u} + (\vec{v} + \vec{w}) &\text{Associative Property}\\
        \text{ii}&. &\vec{u} + \vec{0} &= \vec{u} &\text{Additive Identity Property}\\
        \text{iv}&. &\vec{u} + (-\vec{u}) &= \vec{0} &\text{Additive Inverse Property}\\
        \text{v}&. &r(s\vec{u}) &= (rs)\vec{u} &\text{Associativity of Scalar Multiplication}\\
        \text{vi}&. &(r + s)\vec{u} &= r\vec{u} + s\vec{u} &\text{Distributive Property}\\
        \text{vii}&. &r(\vec{u} + \vec{v}) &= r\vec{u} + r\vec{v} &\text{Distributive Property}\\
        \text{viii}&. &1\vec{u} &= \vec{u} &\text{Identity Property}\\
        \text{ix}&. &0u &= \vec{0} &\text{Zero Property}
    \end{align*}
\end{shaded}
\begin{shaded}
    \noindent\underline{\bf{Proof of Commutative \& Distributive Properties}}
    \vspace{2mm}\\
    Let $\vec{u} = \la x_1, y_1 \ra$ and $\vec{v} = \la x_2, y_2 \ra$
    \vspace{1mm}\\
    \bf{Commutative Property}
    \vspace{1mm}\\
    Apply the commutative property for $\mathbb{R}$
    \begin{align*}
        \vec{u} + \vec{v} = \la x_1 + x_2, y_1 + y_2 \ra = \la x_2 + x_1, y_2 + y_1 \ra = \vec{v} + \vec{u}
    \end{align*}
    \bf{Distributive Property}
    \vspace{1mm}\\
    Apply the distributive property for $\mathbb{R}$
    \begin{align*}
        r(\vec{u} + \vec{v}) &= r \cdot \la x_1 + x_2, y_1 + y_2 \ra\\
        &= \la r(x_1 + x_2), r(y_1 + y_2) \ra\\
        &= \la rx_1 + rx_2, ry_1 + ry_2 \ra\\
        &= \la rx_1, ry_1 \ra + \la rx_2, ry_2 \ra\\
        &= r\vec{u} + r\vec{v}
    \end{align*}
\end{shaded}
\newpage\noindent
In some cases, only the magnitude and direction of a vector are known, not the initial or terminal points. For these vectors, the horizontal and vertical components can be identified using trigonometry.
\begin{center}
    \begin{tikzpicture}[scale=2]
        \draw (0,0) coordinate (a);
        \draw (1,0) coordinate (b);
        \draw (1,1.333) coordinate (c);
        \draw pic["$\theta$",draw=fg1,-,thick,angle eccentricity=1.5,angle radius=5mm]{angle=b--a--c};
        \draw[bg3] (0.9,0)--(0.9,0.1)--(1,0.1);
        \draw[thick,gbblue2] (a)--node[below,color=fg1]{$||\vec{v}||\cos(\theta)$}(b)--node[right,color=fg1]{$||\vec{v}||\sin(\theta)$}(c);
        \draw[->,very thick,-latex,gbred] (a)--node[above,color=fg1,xshift=-0.75em]{$||\vec{v}||$}(c);
    \end{tikzpicture}
\end{center}
Consider the angle $\theta$ formed by the vector $\vec{v}$ and the positive $x$-axis. From the triangle above, the components of the vector $\vec{v}$ are $\la ||\vec{v}||\cos(\theta),||\vec{v}||\sin(\theta) \ra$. Therefore, given an angle and the magnitude of a vector, the components can be found using the sine and cosine of the angle.
\vspace{2mm}\\
\bf{Unit Vectors}
\vspace{2mm}\\
A unit vector is a vector with magnitude 1. For any nonzero vector $\vec{v}$, scalar multiplication can be used to find a unit vector $\vec{u}$ that has the same direction as $\vec{v}$ by multiplying the vector by the reciprocal of its magnitude.
\begin{align*}
    \vec{u} = \frac{1}{||\vec{v}||}\vec{v}
\end{align*} 
Recall that for scalar multiplication, $||k\vec{v}|| = |k| \cdot ||\vec{v}||$. For $\vec{u} = \frac{1}{||\vec{v}||}\vec{v}$ it follows that $||\vec{u}|| = \frac{1}{||\vec{v}||}(||\vec{v}||) = 1$. We say that $\vec{u}$ is the unit vector in the direction of $\vec{v}$. The process of using scalar multiplication to find a unit vector with a given direction is called normalization.
\vspace{1mm}\\
Sometimes it is more convenient to write a vector as a sum of a horizontal vector and a vertical vector. To do this, the standard unit vectors $\ih = \la 1, 0 \ra$ and $\jh = \la 0, 1 \ra$ are used.
\begin{center}
    \begin{tikzpicture}[scale=1.5]
        \draw[->,thick,-latex] (-0.1,0)--(2,0) node[right]{$x$};
        \draw[->,thick,-latex] (0,-0.1)--(0,1.5) node[above]{$y$};
        \draw[<->,very thick,latex-latex,gbblue2] (1,0)--node[below,color=fg1]{$\ih = \la 1, 0 \ra$}(0,0)--node[left,color=fg1]{$\jh = \la 0, 1 \ra$}(0,1);
    \end{tikzpicture}
\end{center}
By applying the properties of vectors, it is possible to express any vector in terms of $\ih$ and $\jh$ in what is called a linear combination.
\begin{align*}
    \vec{v} = \la x, y \ra = \la x, 0 \ra + \la 0, y \ra = x\la 1, 0 \ra + y\la 0, 1 \ra = x\ih + y\jh
\end{align*}
Thus $\vec{v}$ is the sum of a horizontal vector with magnitude $x$ and a vertical vector with magnitude $y$ as shown below.
\begin{center}
    \begin{tikzpicture}[scale=2]
        \draw[->,very thick,-latex,gbblue2] (0,0)--node[below,color=fg1]{$x\ih$}(1.5,0);
        \draw[->,very thick,-latex,gbblue2] (1.5,0)--node[right,color=fg1]{$y\jh$}(1.5,1);
        \draw[->,very thick,-latex,gbblue2] (0,0)--node[above,color=fg1,xshift=-0.25em]{$\vec{v}$}(1.5,1);
    \end{tikzpicture}
\end{center}

\newpage
\subsection{Vectors in Three Dimensions}
The two-dimensional rectangular coordinate system can be expanded by adding a third dimension, the $z$-axis which is perpendicular to both the $x$-axis and the $y$-axis.
\vspace{2mm}\\
\bf{Three-Dimensional Coordinate Systems}
\begin{shaded}
    \noindent\underline{\bf{Definition}}
    \vspace{2mm}\\
    The three-dimensional rectangular coordinate system consists of three perpendicular axes: the $x$-, $y$-, and $z$-axes, and an origin at the point of intersection (0) of the axes. Because each axis is a number line representing all real numbers in $\mathbb{R}$, the three-dimensional system is often denoted by $\mathbb{R}^3$.
\end{shaded}
\noindent In two dimensions, a point in the plane is described with the coordinates $(x,y)$. Each coordinate describes how the point aligns with the corresponding axis. In three dimensions, a new coordinate $z$ is appended to indicate alignment with the $z$-axis: $(x,y,z)$.
\begin{center}
    \begin{tikzpicture}[scale=1.5]
        \draw[<->,thick,latex-latex] (0,2)--(0,0)--(3,0);
        \node[right] at (3,0) {$y$};
        \node[above] at (0,2) {$z$};
        \draw[->,thick,-latex] (0,0)--(-{sqrt(2)},-{sqrt(2)}) node[below,xshift=-0.3em]{$x$};
        \draw[thick,gbblue2] (1.5,0)--(0,0)--(0,1);
        \draw[thick,gbblue2] (0,0)--({-(sqrt(2)/2)},{-(sqrt(2)/2)});
        \draw[thick,dashed,fg4] ({-(sqrt(2)/2)},{-(sqrt(2)/2)})--({-(sqrt(2)/2) + 1.5},{-(sqrt(2)/2)})--({-(sqrt(2)/2) + 1.5},{1-(sqrt(2)/2)})--({-(sqrt(2)/2)},{1-(sqrt(2)/2)})--({-(sqrt(2)/2)},{-(sqrt(2)/2)});
        \draw[thick,dashed,fg4] ({-(sqrt(2)/2) + 1.5},{-(sqrt(2)/2)})--(1.5,0)--(1.5,1)--(0,1)--({-(sqrt(2)/2)},{1-(sqrt(2)/2)});
        \draw[thick,dashed,fg4] ({-(sqrt(2)/2) + 1.5},{1-(sqrt(2)/2)})--(1.5,1);
        \filldraw[fg1] ({1.5 - (sqrt(2))/2},{1 - (sqrt(2))/2}) circle (1.25pt) node[above,xshift=-1.5em,color=fg1]{$(x,y,z)$};
    \end{tikzpicture}
\end{center}
In two-dimensional space, the coordinate plane is defined by a pair of perpendicular axes, these axes allow any location in the plane to be named. In three dimensions, coordinate planes are defined by the coordinate axes as in two dimensions. Each pair of axes forms a coordinate plane: the $xy$-plane, the $xz$-plane, and the $yz$-plane. The $xy$-plane is formally defined as the following set: $\{(x,y,0):x,y\in\mathbb{R}\}$. The $xz$-plane and $yz$-plane are defined as $\{(x,0,z):x,z\in\mathbb{R}\}$ and $\{(0,y,z):y,z\in\mathbb{R}\}$ respectively.
\begin{center}
    \begin{tikzpicture}[line cap=round,line join=round,x={(-0.8562cm,-0.3116cm)},y={(0.5166cm,-0.5166cm)},z={(0cm,0.7975cm)},scale=2]
        \def\op{0.85}
        \draw[fill=gbred,fill opacity=\op] (0,-1,-1)--(0,0,-1)--(0,0,0)--(0,-1,0)--cycle; % It doesn't show
        \draw[fill=gbgreen,fill opacity=\op] (-1,0,-1)--(1,0,-1)--(1,0,0)--(-1,0,0)--cycle;
        \draw[fill=gbred,fill opacity=\op] (0,1,-1)--(0,0,-1)--(0,0,0)--(0,1,0)--cycle;
        \draw[fill=gbblue,fill opacity=\op] (-1,-1, 0)--(-1,1,0)--(1,1,0)--(1,-1,0)--cycle;
        \draw[fill=gbred,fill opacity=\op] (0,-1,0)--(0,0,0)--(0,0,1)--(0,-1,1)--cycle;
        \draw[dashed] (0,0,0)--(0,-1.25,0);
        \draw[fill=gbgreen,fill opacity=\op] (-1,0,0)--(1,0,0)--(1,0,1)--(-1,0,1)--cycle;
        \draw[fill=gbred,fill opacity=\op] (0,1,0)--(0,0,0)--(0,0,1)--(0,1,1)--cycle;
        \draw[->,thick,-latex] (0,0,0)--(1.25,0,0) node[left]{$x$};
        \draw[->,thick,-latex] (0,0,0)--(0,1.25,0) node[below,xshift=1mm]{$y$};
        \draw[->,thick,-latex] (0,0,0)--(0,0,1.25) node[above]{$z$};
        \draw[dashed] (-1,0,0)--(-1.25,0,0);
        \draw[dashed] (0,0,-1)--(0,0,-1.25);
      \end{tikzpicture}
\end{center}
If two points lie in the same coordinate plane, then it is straightforward to calculate the distance between them. The distance $d$ between two points $(x_1,y_1)$ and $(x_2,y_2)$ in the $xy$-coordinate plane is given by the formula
\begin{align*}
    d = \sqrt{(x_2 - x_1)^2 + (y_2 - y_1)^2}
\end{align*}
The formula for the distance between two points in space is a natural extension of this formula.
\begin{shaded}
    \noindent\underline{\bf{The Distance Between Two Points in Space}}
    \vspace{2mm}\\
    The distance $d$ between points $(x_1,y_1,z_1)$ and $(x_2,y_2,z_2)$ is given by the formula
    \begin{align}
        d = \sqrt{(x_2 - x_1)^2 + (y_2 - y_1)^2 + (z_2 - z_1)^2}
    \end{align}
\end{shaded}
\newpage
\noindent\bf{Writing Equations in} $\mathbf{\mathbb{R}^3}$
\vspace{2mm}\\
Compare the graphs of $x = 0$ in $\mathbb{R}$, $\mathbb{R}^2$, and $\mathbb{R}^3$. In the graphs, the same equation describes a point, a line, and a plane.
\begin{center}
    \begin{tikzpicture}[scale=2]
        \draw[->,thick,-latex] (0,0)--(1,0) node{};
        \draw[->,thick,-latex] (0,0)--(-1,0);
        \foreach \x in  {-2,-1,0,1,2} % edit here for the vertical lines
        \draw[shift={(\x/3,0)},color=fg1] (0pt,3pt) -- (0pt,-3pt);
        \foreach \x in {-2,-1,0,1,2} % edit here for the numbers
        \draw[shift={(\x/3,0)},color=fg1] (0pt,0pt) -- (0pt,-3pt) node[below] {$\x$};
        \filldraw[color=gbblue2] (0,0) circle (1.5pt);
        \node at (0,-1.25){$\mathbb{R}$};
    \end{tikzpicture}
    \begin{tikzpicture}[scale=1]
        \draw[step=0.2,bg3] (-1.8,-1.8) grid (1.8,1.8);
        \draw[<->,thick,latex-latex] (0,2)--(0,0)--(2,0);
        \draw[<->,thick,latex-latex] (-2,0)--(0,0)--(0,-2);
        \draw[<->,ultra thick,latex-latex,color=gbblue2] (0,-1.8)--(0,1.8);
        \node[right] at (2,0) {$x$};
        \node[above] at (0,2) {$y$};
        \node at (0,-2.5){$\mathbb{R}^2$};
    \end{tikzpicture}
    \begin{tikzpicture}[x={(-0.8562cm,-0.3116cm)},y={(0.5166cm,-0.5166cm)},z={(0cm,0.7975cm)},scale=1]
        \filldraw[color=gbblue2,fill=gbblue2,opacity=0.8] (2,0,2)--(-2,0,2)--(-2,0,-2)--(2,0,-2)--cycle;
        \draw[->,thick,latex-latex] (-2.25,0,0)--(2.25,0,0);
        \draw[->,thick,latex-latex] (0,-2.5,0)--(0,2.5,0)node[below,xshift=1mm]{$x$};
        \draw[->,thick,latex-latex] (0,0,-2.3)--(0,0,2.3)node[above]{$z$};
        \node[right] at (-2.25,0,0) {$y$};
        \node at (0,0,-3.35){$\mathbb{R}^3$};
    \end{tikzpicture}
\end{center}
In space, the equation $x = 0$ describes all points $(0,y,z)$ and defines the $yz$-plane. Similarly, the the $xy$-plane contains all points of the form $(x,y,0)$ and is defined by the equation $z = 0$. The equation $y = 0$ defines the $xz$-plane.
\begin{center}
    \begin{tikzpicture}[x={(-0.8562cm,-0.3116cm)},y={(0.5166cm,-0.5166cm)},z={(0cm,0.7975cm)},scale=1]
        \filldraw[color=gbgreen2,fill=gbgreen,opacity=0.8] (2,2,0)--(-2,2,0)--(-2,-2,0)--(2,-2,0)--cycle;
        \draw[->,thick,latex-latex] (-2.35,0,0)--(2.35,0,0);
        \draw[->,thick,latex-latex] (0,-2.5,0)--(0,2.5,0)node[below,xshift=1mm]{$x$};
        \draw[->,thick,latex-latex] (0,0,-2.4)--(0,0,2.4)node[above]{$z$};
        \node[right] at (-2.35,0,0) {$y$};
        \node at (0,0,-3.5){$z = 0,\ xy\text{-plane}$};
    \end{tikzpicture}
    \hspace{10mm}
    \begin{tikzpicture}[x={(-0.8562cm,-0.3116cm)},y={(0.5166cm,-0.5166cm)},z={(0cm,0.7975cm)},scale=1]
        \filldraw[color=gbred,fill=gbred,opacity=0.8] (0,2,2)--(0,-2,2)--(0,-2,-2)--(0,2,-2)--cycle;
        \draw[->,thick,latex-latex] (-2.25,0,0)--(2.25,0,0);
        \draw[->,thick,latex-latex] (0,-2.5,0)--(0,2.5,0)node[below,xshift=1mm]{$x$};
        \draw[->,thick,latex-latex] (0,0,-2.4)--(0,0,2.4)node[above]{$z$};
        \node[right] at (-2.25,0,0) {$y$};
        \node at (0,0,-3.5){$y = 0,\ xz\text{-plane}$};
    \end{tikzpicture}
\end{center}
When a plane is parallel to the $xy$-plane, for example, the $z$-coordinate of each point has the same constant value.
\vspace{1mm}\\
A sphere is the set of all points equidistant from a fixed point, the center of the sphere, just as the set of all points in a plane equidistant from the center represents a circle.
\begin{center}
    \begin{tikzpicture}[scale=2]
        \filldraw[fg3] (0,0) circle (1);
        \shade[ball color=gbblue,opacity=0.95] (0,0) circle (1);
        \draw (0,0) circle (1);
        \draw[dashed,fg1] (0,0) ellipse (1 and 0.25);
        \draw[dashed,gbred] (0,0)--node[left,color=fg0]{$r$}({(sqrt(2))/2},{(sqrt(2))/2});
        \filldraw ({(sqrt(2))/2},{(sqrt(2))/2}) circle (1pt);
        \filldraw (0,0) circle (1pt);
    \end{tikzpicture}
\end{center}

\end{document}